% !TEX program = xelatex

\documentclass{article}

\usepackage{titlesec}
\usepackage{titling}
\usepackage[sfdefault]{roboto}
\usepackage[T1]{fontenc}
\usepackage{xcolor}
\usepackage{multicol}
\usepackage[margin=2.5cm]{geometry}

% Secondary color
\definecolor{sec}{gray}{0.2}

\titleformat{\section}
{\Large\bfseries}
{}
{0em}
{}[{\titlerule[1pt]}]

% Horizontal rule command
\newcommand{\hr}{
    \noindent\parbox{\linewidth}{\rule[5pt]{\textwidth}{1pt}}%
}

\renewcommand{\maketitle}{
    \noindent
    {\Large\bfseries\theauthor} \\
    {\large\color{sec}{\thetitle}} \\
    \hr
}

\begin{document}

\author{Guilherme Pontes}
\title{Desenvolvedor de software}

\maketitle
% Address and contact information
\begin{multicols}{2}
    \noindent%
    \begin{flushleft}
        \noindent%
        Endereço \\
        \color{sec}{
            Rua Antônio Rosa Machado \\
            Vila Campo Grande \\
            São Paulo, SP \\
            19 anos
        }
    \end{flushleft}
    \columnbreak
    \begin{flushright}
        \noindent
        Contato \\
        \color{sec}{
            +55 11 99213-9309 \\
            pontes.guisilva@gmail.com \\
            linkedin.com/in/gpontesss \\
            github.com/gpontesss
        }
    \end{flushright}
\end{multicols}

\section{Sum\'ario}

Desenvolvedor de softaware. Interessado em novos conceitos e tecnologias.
Curioso e disposto a fazer todo tipo de experimento. Cozinheiro e contribuidor
para a comunidade open source no tempo livre.

\section{Experi\^{e}ncia}

\begin{multicols}{2}
    \noindent
    \begin{flushleft}
        tembici. \\
        Desenvolvedor backend
    \end{flushleft}
    \columnbreak
    \begin{flushright}
        abril de 2019 até o momento \\
        São Paulo, SP
    \end{flushright}
\end{multicols}

\begin{itemize}
    \item Desenvolvimento de serviços backend com Python e Golang
    \item Migração da aplicação para a arquitetura de microserviços
    \item Interação com bancos de dados PostgreSQL, MySQL e AWS DynamoDB
    \item Infraestrura como código com Terraform
    \item Streaming de dados com GCP PubSub, AWS SQS e Kinesis
    \item Serviços serverless na AWS com Lambdas e API Gateway
    \item Serviços em containers com AWS ECS e ECR
    \item Implementação e configuração de ferramentas de CI/CD (CircleCI/GitHub
    Actions).
\end{itemize}

\begin{multicols}{2}
    \noindent
    \begin{flushleft}
        Zukin Solutions Ltda. \\
        Analista desenvolvedor júnior
    \end{flushleft}
    \columnbreak
    \begin{flushright}
        maio a setembro de 2018 \\
        São Paulo, SP
    \end{flushright}
\end{multicols}

\begin{itemize}
    \item Desenvolvimento de aplicação mobile para aprovação de compras e
    vendas, e gerência de sistema WMS integrado com ERP TOTVS Protheus;
    \item Desenvolvimento de aplicação Web para gerência, localização e
    comunicação com frota de caminhões;
    \item Configuração e manutenção de servidor Debian (GNU/Linux).
\end{itemize}

\section{Habilidades}

\begin{itemize}
    \item Experi\^{e}ncia em projetos com as linguagens JavaScript, HTML, CSS,
    Python e Golang;
    \item Confortável com as linguagens Java e C;
    \item Conhecimento sobre bancos de dado PostgreSQL, MySQL e DynamoDB;
    \item Orquestração de containers com Docker e Docker Compose;
    \item Projeto de serviços utilizando arquitetura REST;
    \item Configuração, manutenção e administração de sistemas Linux;
    \item Gerência de repositórios com Git e GitHub;
    \item Infraestrutura como código com Terraform;
    \item Documentos com {\LaTeX} (inclusive este aqui).
\end{itemize}

\section{Projetos}

\begin{itemize}
    \item Parser de JSON com Golang;
    \item Utilidade em Python para gerar modelos pdf com informações em tabela
    relacional;
    \item Protótipo de um organizador de estoque e interface gráfica para
    controle em Python;
    \item Exemplo de stack para agregação de logs com fluentd, kibana e
    elasticsearch (EFK) para aplicações web.
\end{itemize}

\section{Formação}

\begin{multicols}{2}
    \noindent
    \begin{flushleft}
        ETEC Takashi Morita \\
        Técnico em Automação Industrial
    \end{flushleft}
    \columnbreak
    \begin{flushright}
        jan. de 2015 a dez. de 2017 \\
        São Paulo, SP
    \end{flushright}
\end{multicols}


\end{document}
